\documentclass{article}

\usepackage{amsmath}
\usepackage{amsfonts}
\usepackage{polski}
\usepackage[polish]{babel}
\usepackage[hidelinks]{hyperref}

\begin{document}
\author{
    Antoni Kucharski \\
    Piotr Rusak \\
    Paweł Prus \\
    Kacper Garus
}
\title{
    \Huge \textbf{Badania Operacyjne} \\
    \textbf{Informatyka 2024/2025} \\
}

\begin{titlepage}
\maketitle    
\end{titlepage}

\tableofcontents
\newpage

\section{Wstęp}

Celem projektu jest analiza i implementacja algorytmu mrówkowego do rozwiązania problemu znalezienia najlepszej
trasy przejazdu z przystanku A do przystanku B w sieci linii komunikacyjnych. Szukanie trasy uwzględnia
czasy odjazdu pojazdów z przystanków, czas przejazdu pomiędzy przystankami i możliwość przesiadek.
Użytkownik może podać czas, o której chce rozpocząć podróż jak i przystanek początkowy i końcowy.

\section{Opis zagadnienia}

\subsection{Model matematyczny}
\subsubsection{Mapa}
Mapa komunikacyjna jest grafem nieskierowanym \(G = (V, E)\), gdzie \(V\subset{\mathbb{N}}^2\) to zbiór wierzchołków
reprezentujących punkty na mapie jako para współrzędnych \((x, y)\), a \(E \subset \{\{u, v\} : u, v\in V\}\) to zbiór krawędzi. Ponadto
zachodzi zależność
\[\forall\;\left\{(x_u, y_u), (x_v, y_v)\right\} \in E\;\; |x_u - x_v| + |y_u - y_v| = 1\]
To oznacza, że sąsiednie wierzchołki różnią się dokładnie o jedną współrzędną, co odpowiada ruchowi w górę, w dół, w lewo lub w prawo.

\subsubsection{Zbiór linii komunikacyjnych}
Każda linia komunikacyjna jest opisana jako krotka \((n, S)\),
gdzie \(n\) to numer linii, a \(S = \{(v, f) : v\in V, f\in\{0, 1\}\}\) to zbiór par wierzchołków i flagi określające czy dany punkt jest przystankiem czy nie.

\subsubsection{Harmonogramy odjazdu}
Dla danego przystanku \(v\) definiujemy zbiór \[H_v = \left\{(n_0, h_0), (n_1, h_1), \dots, (n_n, h_n)\right\}\]
określającą harmonogram odjazdu, gdzie \(n\) oznacza liczbę odjazdów z przystanku \(v\),
\(l_i\) to numer linii komunikacyjnej, a \(h_i\) to czas odjazdu tej linii.

\subsubsection{Funkcja trasy}
Jeśli przez \(N\) oznaczymy liczbę tras z z przystanku \(v_0\) do przystanku \(v_m\), to \(k\)-tą z nich definiujemy jako sekwencję
\[P_k = \left\{(v_0, v_1^k, n_0^k), (v_1^k, v_2^k, n_1^k), \ldots, (v_{m - 1}^k, v_{m}, n_{m - 1}^k)\right\}\]
gdzie \(m\) to liczba przystanków na trasie, \(v_0, v_1^k, \dots, v_{m - 1}^k v_m\) to kolejne przystanki, a \(n_i^k\) to linia komunikacyjna, którą jedziemy z przystanku \(v_i\) do przystanku \(v_{i + 1}\).
Definiujemy funkcję
\[f(v_0, v_m) = \left\{P_1, P_2, \dots, P_N\right\}\]
zwracającą zbiór wszystkich tras z przystanku \(v_0\) do przystanku \(v_m\) rozpoczynających się w chwili \(t_0\).

\subsubsection{Funkcja celu}
Niech \(t(n, v, t_0) = \min(\{h - t_0 : (n, h)\in H_v, h \geq t_0\})\) oznacza czas oczekiwania na linię \(n\) będąc na przystanku \(v\)
w chwili \(t_0\), a \(d(n, u, v)\) oznacza czas przejazdu linią \(n\) z przystanku \(u\) do przystanku \(v\). Wtedy funkcja celu
dla danej trasy jest zdefiniowana jako
\[
    T(P, t_0) = \sum_{i = 0}^{m - 1}(t(n_i, v_i, t_i) + d(n_i, v_i, v_{i + 1}))
\]
gdzie \(t_i\) to czas przyjazdu na przystanek \(v_i\) oraz \[P = \{(v_0, v_1, n_0),\dots,(v_{m - 1}, v_m, n_{m - 1})\}\in f(v_0, v_m)\]

\subsection{Szukana wartość}
Szukana przez nas wartość to trasa \(P_{opt}\) z przystanku \(v_0\) do przystanku \(v_m\) zaczynając o godzinie \(t_0\)
o minimalnej wartości funkcji celu czyli
\[P_{opt} = \operatorname*{\arg\,\min}_{P\in f(v_0, v_m)} T(P, t_0)\]

\section{Opis algorytmu}

\section{Aplikacja}

\section{Eksperymenty}

\section{Podsumowanie}

\end{document}